\documentclass[a4paper, 11pt]{article}
\usepackage[italian]{babel}
\usepackage[utf8x]{inputenc}
\usepackage{lmodern}

\font\myfont=cmr12 at 25pt
\newcommand{\code}{\texttt}

\title{\myfont Algoritmi genetici v.0}
\author{Alessio Marchetti}
\date{}

\begin{document}
\maketitle

\section{Introduzione}

Gli algoritmi genetici (GA) sono algoritmi di ricerca basati sulle meccaniche
della selezione naturale e della genetica. Essi si basano sulla manipolazione di
una popolazione di individui, ciascuno identificato da una propria stringa, che
ne definisce il comportamento. Tale stringa può essere assimilata al DNA. Sulla
popolazione vengono essenzialmente eseguiti tre tipi di azioni:
\begin{enumerate}
    \item
    \textit{riproduzione}: Da una generazione (ovvero una popolazione in un
        certo istante) si selezionano gli individui più adatti alla
        sopravvivenza e si portano alla generazione successiva.
    \item
    \textit{crossover}: Gli individui di una popolazione scambiano fra di loro
        porzioni di DNA.
    \item
    \textit{mutazione}: Sezioni di DNA variano casualmente, con una frequenza
        fissata generalmente molto bassa.
\end{enumerate}



\section{Simulazione ``a mano''}

Per avere un'idea del funzionamento di un GA mostro un esempio di funzionamento
molto semplice e di cui è ben nota una soluzione. Si tratta di trovare il
massimo e il punto di massimo della funzione $f(x)=x^2$ nell'intervallo
$[0,31]$.

Come prima cosa è necessario definire un vocabolario che andrà a comporre le
stringhe del DNA. Lo faccio nella maniera più semplice possibile, ovvero
$$V=\{0,1\}$$

Inoltre definisco un individuo come un elemento dell'insieme $V^5$. Per esempio
è un individuo \code{01001}. Ogni individuo codifica un certo valore di $x$, che
per comodità scelgo essere la sua rappresentazione in base decimale. Il valore
corrispondente a quello scelto in precedenza sarà dunque $9$.

Per l'implementazione di un GA è inoltre necessaria una funzione detta di
\textit{fitness}, tale che, dato un individuo, ne indichi la sua idoneità.
Nell'esempio preso in esame è spontaneo scegliere $f$ stessa. Quindi
\code{01001}  ha un fitness di $f(9)=9^2=81$.

In questo esempio scelgo di avere una popolazione composta da $N=4$ individui. In
genere questo numero è molto più grande per avere un algoritmo efficiente. La
popolazione iniziale è generata casualmente, ovvero ogni lettera di ogni stringa
è il risultato di un lancio di moneta. Ottengo la seguente popolazione, con
relativo fitness.

\begin{table}[h!]
\begin{tabular}{lllll}
\multicolumn{1}{l|}{n} & Stringa      & Valore $x$ & fitness &  \\ \cline{1-4}
\multicolumn{1}{l|}{1} & \code{01101} & 13         & 169     &  \\
\multicolumn{1}{l|}{2} & \code{11000} & 24         & 576     &  \\
\multicolumn{1}{l|}{3} & \code{01000} & 8          & 64      &  \\
\multicolumn{1}{l|}{4} & \code{10011} & 19         & 361     &  \\ \cline{1-4}
\multicolumn{3}{c}{totale}                         & 1170    & 
\end{tabular}
\end{table}



\subsection{Riproduzione}

A questo punto voglio generare una nuova generazione a partire dagli individui
con fitness maggiore. Per fare ciò ad ogni individuo $i_k$  assegno la
probabilità di riproduzione 
$$p_k=\frac{f(k)}{\sum\limits_{j}f(j)}$$
dove al numeratore compare il fitness di tale individuo e a denominatore la
somma di tutti i fitness. Risulta chiaro che a maggiore fitness corrisponde
maggiore probabilità di riproduzione e che la somma di tutti i $p_k$ valga $1$.
La tabella risulta dunque essere

\begin{table}[h!]
\begin{tabular}{lllll}
\multicolumn{1}{l|}{$k$} & Stringa        & Valore $x$ & fitness & $p_k$ \\ \hline
\multicolumn{1}{l|}{1}   & \code{01101} & 13         & 169     & 0.14  \\
\multicolumn{1}{l|}{2}   & \code{11000} & 24         & 576     & 0.49  \\
\multicolumn{1}{l|}{3}   & \code{01000} & 8          & 64      & 0.05  \\
\multicolumn{1}{l|}{4}   & \code{10011} & 19         & 361     & 0.31  \\ \hline
\multicolumn{3}{c}{totale}                           & 1170    & 1.00 
\end{tabular}
\end{table}

Una volta determinate le probabilità, scelgo casualmente gli individui per la
nuova popolazione. Ciascuno di essi ha probabilità $p_k$ di essere uguale
all'individuo $i_k$. Su una popolazione di $N$ individui ci aspettiamo quindi di
avere $Np_k$ individui uguali a $i_k$.
Se si definisce il fitness medio 
$$\overline{f}=\frac{\sum\limits_{j}f(j)}{N}$$
si ha che 
$$Np_k=N \frac{f(k)}{\sum\limits_{j}f(j)}=\frac{f(k)}{\overline{f}}$$
Questo significa che se un certo individuo ha un fitness superiore alla media,
ovvero $f>\overline{f}$, avrà $f/\overline{f}>1$, cioè tenderà ad aumentare il
suo numero di copie nella generazione successiva. Analogamente un individuo con
un fitness inferiore alla media tenderà a diminuire il proprio numero di copie.

Vado dunque a eseguire la riproduzione sulla popolazione in esame, ottenendo i
seguenti risultati.

\begin{table}[h!]
\begin{tabular}{lcccccl}
\multicolumn{1}{l|}{$k$} & \multicolumn{1}{l}{Stringa} & \multicolumn{1}{l}{Valore $x$} & \multicolumn{1}{l}{fitness} & \multicolumn{1}{l}{$p_k$} & \multicolumn{1}{l}{$f/\overline{f}$} & \begin{tabular}[c]{@{}l@{}}numero di individui\\ nella nuova generazione\end{tabular} \\ \hline
\multicolumn{1}{l|}{1} & \code{01101} & 13 & 169 & 0.14 & 0.58 & \multicolumn{1}{c}{1} \\
\multicolumn{1}{l|}{2} & \code{11000} & 24 & 576 & 0.49 & 1.96 & \multicolumn{1}{c}{2} \\
\multicolumn{1}{l|}{3} & \code{01000} & 8 & 64 & 0.05 & 0.21 & \multicolumn{1}{c}{0} \\
\multicolumn{1}{l|}{4} & \code{10011} & 19 & 361 & 0.31 & 1.23 & \multicolumn{1}{c}{1} \\ \hline
\multicolumn{3}{c}{totale} & 1170 & 1.00 & 4.00 &  \\
\multicolumn{3}{c}{media} & 229.5 & 0.25 & 1.00 &  \\
\multicolumn{3}{c}{massimo} & 576 & 0.49 & 1.96 & 
\end{tabular}
\end{table}



\subsection{Crossover}
Supponiamo in una certa popolazione di avere individui con stringhe composte da
un numero $l$ di lettere tratte dal vocabolario $V$. Nel nostro esempio $l=5$.
Indico con $i_k^j$ la $j$-esima lettera della stringa dell'individuo $i_k$.
Vediamo cosa succede eseguendo un crossover su due individui $i_1$ e $i_2$.
Scelgo casualmente un un intero $c$ compreso tra $1$ e $l-1$. Il risultato del
crossover sono un individuo $i_1'$

\end{document}

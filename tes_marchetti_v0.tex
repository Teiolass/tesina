\documentclass[a4paper, 11pt]{article}
\usepackage[italian]{babel}
\usepackage[utf8x]{inputenc}
\usepackage{lmodern}
\usepackage{amsmath}

\font\myfont=cmr12 at 25pt
\newcommand{\code}{\texttt}

\title{\myfont Algoritmi genetici v.0}
\author{Alessio Marchetti}
\date{}

\begin{document}
\maketitle

\section{Introduzione}

Gli algoritmi genetici (GA) sono algoritmi di ricerca basati sulle meccaniche
della selezione naturale e della genetica. Essi si basano sulla manipolazione di
una popolazione di individui, ciascuno identificato da una propria stringa, che
ne definisce il comportamento. Tale stringa può essere assimilata al DNA. Sulla
popolazione vengono essenzialmente eseguiti tre tipi di azioni:
\begin{enumerate}
    \item
    \textit{riproduzione}: Da una generazione (ovvero una popolazione in un
        certo istante) si selezionano gli individui più adatti alla
        sopravvivenza e si portano alla generazione successiva.
    \item
    \textit{crossover}: Gli individui di una popolazione scambiano fra di loro
        porzioni di DNA.
    \item
    \textit{mutazione}: Sezioni di DNA variano casualmente, con una frequenza
        fissata generalmente molto bassa.
\end{enumerate}



\section{Simulazione ``a mano''}

Per avere un'idea del funzionamento di un GA mostro un esempio di funzionamento
molto semplice e di cui è ben nota una soluzione. Si tratta di trovare il
massimo e il punto di massimo della funzione $f(x)=x^2$ nell'intervallo
$[0,31]$.

Come prima cosa è necessario definire un vocabolario che andrà a comporre le
stringhe del DNA. Lo faccio nella maniera più semplice possibile, ovvero
$$V=\{0,1\}$$

Inoltre definisco un individuo come un elemento dell'insieme $V^5$. Per esempio
è un individuo \code{01001}. Ogni individuo codifica un certo valore di $x$, che
per comodità scelgo essere la sua rappresentazione in base decimale. Il valore
corrispondente a quello scelto in precedenza sarà dunque $9$.

Per l'implementazione di un GA è inoltre necessaria una funzione detta di
\textit{fitness}, tale che, dato un individuo, ne indichi la sua idoneità.
Nell'esempio preso in esame è spontaneo scegliere $f$ stessa. Quindi
\code{01001}  ha un fitness di $f(9)=9^2=81$.

In questo esempio scelgo di avere una popolazione composta da $N=4$ individui. In
genere questo numero è molto più grande per avere un algoritmo efficiente. La
popolazione iniziale è generata casualmente, ovvero ogni lettera di ogni stringa
è il risultato di un lancio di moneta. Ottengo la seguente popolazione, con
relativo fitness.

\begin{table}[h!]
\begin{tabular}{lllll}
\multicolumn{1}{l|}{k} & Stringa      & Valore $x$ & fitness &  \\ \cline{1-4}
\multicolumn{1}{l|}{1} & \code{01101} & 13         & 169     &  \\
\multicolumn{1}{l|}{2} & \code{11000} & 24         & 576     &  \\
\multicolumn{1}{l|}{3} & \code{01000} & 8          & 64      &  \\
\multicolumn{1}{l|}{4} & \code{10011} & 19         & 361     &  \\ \cline{1-4}
\multicolumn{3}{c}{totale}                         & 1170    & 
\end{tabular}
\end{table}



\subsection{Riproduzione}

A questo punto voglio generare una nuova generazione a partire dagli individui
con fitness maggiore. Per fare ciò ad ogni individuo $i_k$  assegno la
probabilità di riproduzione 
$$p_k=\frac{f(i_k)}{\sum\limits_{j}f(i_j)}$$
dove al numeratore compare il fitness di tale individuo e a denominatore la
somma di tutti i fitness. Risulta chiaro che a maggiore fitness corrisponde
maggiore probabilità di riproduzione e che la somma di tutti i $p_k$ valga $1$.
La tabella risulta dunque essere

\begin{table}[h!]
\begin{tabular}{lllll}
\multicolumn{1}{l|}{$k$} & Stringa        & Valore $x$ & fitness & $p_k$ \\ \hline
\multicolumn{1}{l|}{1}   & \code{01101} & 13         & 169     & 0.14  \\
\multicolumn{1}{l|}{2}   & \code{11000} & 24         & 576     & 0.49  \\
\multicolumn{1}{l|}{3}   & \code{01000} & 8          & 64      & 0.05  \\
\multicolumn{1}{l|}{4}   & \code{10011} & 19         & 361     & 0.31  \\ \hline
\multicolumn{3}{c}{totale}                           & 1170    & 1.00 
\end{tabular}
\end{table}

Una volta determinate le probabilità, scelgo casualmente gli individui per la
nuova popolazione. Ciascuno di essi ha probabilità $p_k$ di essere uguale
all'individuo $i_k$. Su una popolazione di $N$ individui ci aspettiamo quindi di
avere $Np_k$ individui uguali a $i_k$.
Se si definisce il fitness medio 
$$\overline{f}=\frac{\sum\limits_{j}f(i_j)}{N}$$
si ha che 
$$Np_k=N \frac{f(i_k)}{\sum\limits_{j}f(i_j)}=\frac{f(i_k)}{\overline{f}}$$
Questo significa che se un certo individuo ha un fitness superiore alla media,
ovvero $f>\overline{f}$, avrà $f/\overline{f}>1$, cioè tenderà ad aumentare il
suo numero di copie nella generazione successiva. Analogamente un individuo con
un fitness inferiore alla media tenderà a diminuire il proprio numero di copie.

Vado dunque a eseguire la riproduzione sulla popolazione in esame, ottenendo i
seguenti risultati.

\begin{table}[h!]
\begin{tabular}{lcccccl}
\multicolumn{1}{l|}{$k$} & \multicolumn{1}{l}{Stringa} & \multicolumn{1}{l}{Valore $x$} & \multicolumn{1}{l}{fitness} & \multicolumn{1}{l}{$p_k$} & \multicolumn{1}{l}{$f/\overline{f}$} & \begin{tabular}[c]{@{}l@{}}numero di individui\\ nella nuova generazione\end{tabular} \\ \hline
\multicolumn{1}{l|}{1} & \code{01101} & 13 & 169 & 0.14 & 0.58 & \multicolumn{1}{c}{1} \\
\multicolumn{1}{l|}{2} & \code{11000} & 24 & 576 & 0.49 & 1.96 & \multicolumn{1}{c}{2} \\
\multicolumn{1}{l|}{3} & \code{01000} & 8 & 64 & 0.05 & 0.21 & \multicolumn{1}{c}{0} \\
\multicolumn{1}{l|}{4} & \code{10011} & 19 & 361 & 0.31 & 1.23 & \multicolumn{1}{c}{1} \\ \hline
\multicolumn{3}{c}{totale} & 1170 & 1.00 & 4.00 &  \\
\multicolumn{3}{c}{media} & 229.5 & 0.25 & 1.00 &  \\
\multicolumn{3}{c}{massimo} & 576 & 0.49 & 1.96 & 
\end{tabular}
\end{table}



\subsection{Crossover}

Supponiamo in una certa popolazione di avere individui con stringhe composte da
un numero $l$ di lettere tratte dal vocabolario $V$. Nel nostro esempio $l=5$.
Indico con $i_k^j$ la $j$-esima lettera della stringa dell'individuo $i_k$.
Vediamo cosa succede eseguendo un crossover su due individui $i_1$ e $i_2$.
Scelgo casualmente un un intero $c$ compreso tra $1$ e $l-1$ detto sito di
crossover. Il risultato del crossover è un individuo $i_1'$ tale che
$i_1'^n=i_1^n$ se $n\leq c$ e $i_1'^n=i_2^n$ altrimenti. Allo stesso modo viene
prodotto un individuo $i_2'$ tale che $i_1'^n=i_2^n$ se $n\leq c$ e
$i_1'^n=i_1^n$ altrimenti. In altri termini da un certo punto in poi le due
stringhe si scambiano tra di loro. Per eseguire un crossover su un'intera
popolazione, si devono accoppiare casualmente gli individui, e poi eseguire il
crossover su ciascuna coppia. 

Partiamo dalla popolazione derivante dalla riproduzione e accoppiamo casualmente
gli individui. Scegliamo inoltre casualmente il sito di crossover.
\begin{table}[h!]
\begin{tabular}{l|ccc}
$k$ & Stringa & Compagno & Sito di crossover \\ \hline
1 & \code{01101} & 2 & 4 \\
2 & \code{11000} & 1 & 4 \\
3 & \code{11000} & 4 & 2 \\
4 & \code{10011} & 3 & 2
\end{tabular}
\end{table}

Andiamo a studiare la prima coppia. La prima stringa si spezza come
\code{0110-1}, mentre la seconda come \code{1100-0}. Si otterranno dunque i
nuovi individui \code{01100} e \code{11001}. Si procede in modo analogo sulla
seconda coppia per ottenere \code{11011} e \code{10000}.



\subsection{Mutazione}

Durante la mutazione ciascuna lettera di ogni stringa ha un probabilità di
$\varepsilon=0.001$ di diventare un'altra lettera del vocabolario. In questo
caso ci aspettiamo $Nl \varepsilon=4\cdot 5 \cdot 0.001= 0.02$ mutazioni. Di
fatto simulando non ne troviamo alcuna. 



\subsection{Analisi dei risultati}

Andiamo infine ad analizzare quali sono stati i risultati delle tre operazioni.
Nella tabella seguente sono presenti alcuni dati sulla seconda generazione
comparati con la prima.
\begin{table}[h!]
\begin{tabular}{cccc|c}
\multicolumn{1}{c|}{$k$} & Stringa & Valore $x$ & fitness gen. 2 & fitness gen. 1 \\ \hline
\multicolumn{1}{c|}{1} & \code{01100} & 12 & 144 & 169 \\
\multicolumn{1}{c|}{2} & \code{11001} & 25 & 625 & 576 \\
\multicolumn{1}{c|}{3} & \code{11011} & 27 & 729 & 64 \\
\multicolumn{1}{c|}{4} & \code{10000} & 16 & 256 & 361 \\ \hline
\multicolumn{3}{c}{media} & 438.5 & 229.5 \\
\multicolumn{3}{c}{massimo} & 625 & 576
\end{tabular}
\end{table}

Risulta dunque evidente che si ha avuto un incremento del fitness sia nel valore
massimo che nel valore medio.




\section{Fondamenti matematici}

Una grossa domanda che è necessario porsi per comprendere il perché del
funzionamento dei GA è la seguente: quali informazioni porta una certa
popolazione? Ovviamente una risposta è data dagli individui stessi, il cui
fitness cresce con l'avanzare delle generazioni. Un secondo punto di vista si
può trovare considerando le similitudini tra le stringhe della popolazione. In
altre parole il miglioramento generale della popolazione tra una generazione e
l'altra è parallelo all'accentuarsi di alcune similitudini. Per studiare questo
processo si fa uso della nozione di schema.


\subsection{Schemi}

Consideriamo sempre una poplazione di $N$ individui caratterizzati da stringhe di
lunghezza $l$ composte da lettere del vocabolario $V$. Scelgo un carattere
jolly, che sarà identificato dal simbolo \code{*}. A questo si può costruire un
vocabolario esteso $V_+=V \cup \{\code{*}\}$. Uno schema $h$ è un elemento
dell'insieme $V_+^l$. Proseguendo nell'esempio di prima, è uno schema
\code{10*1*}. Si può stabilire una corrispondenza tra uno schema e un insieme
di possibili stringhe, tale che ogni lettera dello schema diversa da \code{*}
sia uguale a quella corrispondente nella stringa. Dunque allo schema
\code{10*1*} saranno associate le stringhe \code{10010}, \code{10011},
\code{10110} e \code{10111}. Queste stringhe prendono il nome di rappresentanti
dello schema.

Definisco ora due funzioni che saranno utili in seguito. La prima \`e l'ordine,
che si denota con $o(h)$ ed \`e il numero di caratteri fissi (cio\`e diversi dal
carattere jolly) nello schema. Quindi $o(\code{110*1**1})=5$. La seconda
funzione \`e detta lunghezza caratteristica di uno schema ed \`e la distanza tra
la prima lettera fissa e l'ultima. Si scrive come $\delta(h)$. Per esempio
$\delta( \code{10*1**)} = 3$. Infatti il primo carattere fisso si trova in
posizione $1$ e l'ultimo in posizione $4$.  Dunque il valore che cerchiamo \`e
$4-1=3$.

Inoltre possiamo estendere la funzione di fitness agli schemi. Sia $h$ uno
schema e $i_1, i_2, \ldots ,i_n$ suoi rappresentanti. allora il fitness medio di
$h$ vale $$f(h)=\frac{\sum\limits_{j=1}^n f(i_j)}{n}$$



\subsection{Schemi e riproduzione}

Chiamiamo $m(h, t)$ il numero di rappresentanti di uno schema $h$ nella
generazione $t$. Parlando del fitness medio rispetto ai rappresentanti presenti
nella popolazione, chiamati $r_1, r_2, \ldots, r_n$ con $n=m(h,t)$, ci
aspettiamo che il numero di copie di $r_j$ sia 
$$\frac{f(r_j)}{\overline{f}}$$
e che dunque
\begin{align*}
m(h,t+1) =
\frac{f(r_1)}{\overline{f}}
+\frac{f(r_2)}{\overline{f}}+ \cdots +
\frac{f(r_n)}{\overline{f}} =\\
\frac{\sum\limits_{j=1}^n f(r_j)}{\overline{f}} =
\frac{n}{\overline{f}} \frac{\sum\limits_{j=1}^n f(r_j)} {n} = n \frac{f(h)}{\overline{f}} = m(h,t)\frac{f(h)}{\overline{f}}
\end{align*}
Possiamo scegliere una certa $\lambda$ tale che $f(h)=(1+\lambda
)\overline{f}$. Tale valore indica quanto il fitness di uno schema si discosta
dalla media, e sar\`a positivo se $h$ \`e un buono schema e negativo
altrimenti. L'equazione precedente diventa dunque
$$m(h,t+1)=m(h,t)\frac{(1+\lambda )\overline{f}}{\overline{f}}=(1+\lambda
)m(h,t)$$
Assumiamo che $\lambda$ non vari significativamente, ovvero consideriamo uno
schema che rimane sempre ``buono'' oppure sempre ``cattivo''. In tal caso si
avr\`a
$$m(h,t)=(1+\lambda)^tm(h,0)$$
Ovvero il numero di rappresentanti di uno schema crescer\`a o decrescer\`a con
regime esponenziale.



\subsection{Schemi e crossover}

Occupiamoci ora di come il crossover influisca sul numero di rappresentanti di
un certo schema. Prendiamo come esempio gli schemi
\begin{align*}
h_1=\code{*1****0}\\
h_2=\code{***10**}
\end{align*}
di cui l'individuo $i=\code{0111000}$ \`e un rappresentante. Risulta evidente
che \`e molto pi\`u facile che un crossover distrugga $h_1$ piuttosto che
$h_2$. Infatti il primo schema viene distrutto dai crossover su $i$ con sito
$c=2,3,4,5,6$. Per il secondo si pu\`o avere solo per $c=4$. Risulta evidente
che il numero di crossover ``distruttivi'' \`e $\delta (h)$. Il numero dei
possibili \`e ovviamente $l-1$ con $l$ lunghezza della stringa. Dunque la
probabilit\`a che uno schema venga salvato da un crossover \`e
$$p_\text{s}=1-\frac{\delta (h)}{l-1}$$

Nel caso in cui un crossover distruttivo avvenisse tra due rappresentanti dello
schema, lo schema rimarrebbe comunque intatto. Ne consegue che quello trovato
prima \`e solo un limite inferiore. Si ha dunque che il contributo relativo al
crossover al numero di rappresentanti di $h$ \`e
$$m(h, t+1)\geq p_\text{s}m(h,t)=m(h,t)\left(1- \frac{\delta (h)}{l-1} \right)$$


\subsection{Schemi e mutazione}

Studiamo il comportamento degli schemi con una mutazione con frequenza
$\varepsilon$, ovvero ogni carattere ha la probailit\`a di modificarsi pari a
$\varepsilon$. Risulta dunque evidente che la probabilit\`a di uno schema di
resistere ad una mutazione \`e pari a 
$$p_\text{r}=\left(1-\varepsilon\right)^{o(h)}$$
Siccome stiamo trattando valori di $\varepsilon \ll 1$, possiamo sviluppare al
primo ordine per ottenere
$$p_\text{r}=1-\varepsilon o(h)$$
e dunque considerando solo il contributo della mutazione si ottiene
$$m(h,t+1)=m(h,t)\left(1-\varepsilon o(h)\right)$$


\subsection{Teorema degli schemi}

Non rimane che combinare le equazioni per ottenere che
$$m(h, t+1) \geq m(h,t) \left(1+\lambda \right) \left(1-\frac{\delta
(h)}{l-1}\right) \left(1- \varepsilon o(h) \right)$$
Ci\`o significa che schemi corti, con piccola lunghezza caratteristica e con un
fitness sempre superiore alla media, tenderanno ad aumentare il numero di propri
rappresentanti con andamento esponenziale. Questo risultato \`e noto come
teorema degli schemi o teorema fondamentale degli algoritmi genetici. Infatti la
vera potenza dei GA sta nel riuscire a processare parallelamente un grande numero
di schemi.


\section{Importanza della diversit\`a tra individui}

Consideriamo due generazioni consecutive $t$ e $t+1$. Partendo da condizioni
casuali (come nell'esempio) e procedendo a svolgere le varie operazioni,
chiamiamo $p(i,t)$ la probabilit\`a di trovare l'individuo $i$ nella
generazione $t$. Ovviamente su una popolazione di $N$ individui, ci aspettiamo
$Np(i,t)$ individui uguali a $i$. Per la legge della riproduzione
$$p(i, t+1) = p(i,t) \frac{f(i)}{\overline{f}(t)}$$
con $\overline{f}(t)$ fitness medio nella generazione $t$. Inoltre possiamo
risrivere il fitness medio di una popolazione nei termini dei $p(i_j,t)$. Si ha
infatti
$$\overline{f}(t)=\sum\limits_j p(i_j)f(i_j)$$


Dimostriamo per induzione che $\sum_j p(i_j, t)=1$. Ovviamente il caso base \`e
facile e si ha 
$$p(i_j,0)=\frac{1}{|V|^l}$$
e siccome i possibili $i_j$ sono in tutto $|V|^l$ la somma vale $1$.
Supponiamo allora che la somma valga $1$ per una certa generazione $t$. Allora
$$\sum\limits_j p(i_j,t+1) = 
\sum\limits_j p(i_j)\frac{f(i_j)}{\overline{f}(t)} = 
\frac{\sum\limits_j p(i_j)f(i_j)}{\overline{f}(t)}=
\frac{\overline{f}(t)}{\overline{f}(t)}=1$$

Definiamo la varianza come 
$$\sigma^2(t) = \sum\limits_j p(i_j, t) \left[f(i_j)- \overline{f}(t)\right]^2$$
dove la sommatoria \`e intesa su tutti i possibili individui. Tanto maggiore \`e
la varianza tanto \`e maggiore la diversit\`a tra gli individui nella
popolazione. Si ha che 
\begin{align*}
\sigma^2(t)=
    &\sum\limits_j p(i_j,t)\left[f^2(i_j) - 2f(i_j)\overline{f}
    (t)+\overline{f}^2(t)\right]=\\
    &\cdots\\
    &\code{...Passaggi algebrici...}\\
    &\cdots\\
    &\sum\limits_j p(i_j,t)\left[f^2(i_j)-f(i_j)\overline{f}(t)\right]
\end{align*}

Definiamo inoltre la risposta alla riproduzione
$$R(t) = \overline{f}(t+1)-\overline{f}(t)$$
che indica di quanto la riproduzione riesca a far crescere il fitness medio.
Dunque
\begin{align*}
R(t)=\sum\limits_j \left[p(i_j,t+1)-p(i_j,t)\right]f(i_j)=\\
    \code{passggi algebrici}\\
    =\frac{1}{\overline{f}(t)} \sum\limits_j p(i_j,t)
    \left[f^2(i_j)-f^2(i_j)\overline{f}(t)\right]=\\
    \frac{\sigma^2(t)}{\overline{f}(t)}
\end{align*}

Ne consegue che si ha un aumento maggiore nel fitness di una popolazione quando
la varianza \`e grande, e cio\`e quando gli individui sono diversi tra di loro.
A questo fine sono infatti volte le operazioni di crossover e mutazione.



\end{document}

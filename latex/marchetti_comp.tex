\documentclass[a4paper, 11pt, onecolumn]{article}
\usepackage[italian]{babel}
\usepackage[utf8x]{inputenc}
\usepackage{lmodern}
\usepackage{amsmath}
\usepackage[dvipsnames,usenames]{color}

\font\myfont=cmr12 at 25pt
\definecolor{mgray}{gray}{0.3}
\newcommand{\code}{\texttt}
\newcommand*{\MyMarginNoteFormat}{%
    \scriptsize \bfseries \leavevmode \color{mgray}%
}
\newcommand{\margin}[1]{%
    \marginpar
        [\raggedleft  \MyMarginNoteFormat #1]%
        {\raggedright \MyMarginNoteFormat #1}%
}
\newcommand{\slide}{\margin{Slide}}

\title{\myfont Tesina - Algoritmi genetici}
\author{Alessio Marchetti}
\date{}

\begin{document}
\maketitle

\textit{\small \textbf{Nota:} Le note a margine indicano quando cambiare la
slide. Tuttavia alcune slides sono frammentate in pi\`u fasi, e necessitano di
andare avanti anche se non specificato. \vspace{7mm}}

L'argomento della mia tesina \`e uno studio del funzionamento degli algoritmi
genetici. Gli algoritmi genetici sono innanzi tutto algoritmi, ovvero un insieme
di istruzioni utili a risolvere determinate classi di problemi. \slide
L'aggettivo ``genetici'' viene attribuito in quanto i processi che studieremo
attuano meccaniche ispirate ai meccanismi della natura, in particolare quelli
della genetica e della selezione naturale.

La metafora si struttura in questo modo: dato un certo problema, l'obiettivo
dell'algoritmo \`e quello di trovare la soluzione migliore. Allora si \slide
genera in modo casuale una popolazione di candidate soluzioni e di queste si
scelgono le pi\`u adatte a risolvere il problema. Poi  a partire da esse si
costruiscono nuove candidate in modo tale da avere una popolazione mediamente
migliore. Quest'ultimo processo ricalca la riproduzione sessuata, infatti da
coppie di soluzioni, vengono prodotte soluzioni figlie con caratteristiche
comuni ai due genitori.

Per spiegare meglio il funzionamento operativo di un algoritmo genetico, \slide
seguiremo passo a passo un caso specifico. Il problema di cui tratter\`o \`e di
carattere matematico, ovvero la ricerca del massimo e del punto di massimo di
una funzione in un  determinato intervallo chiuso. Ovvero, dato un grafico, il
compito \`e quello di cercarne il punto pi\`u alto. La funzione che ho scelto
\`e molto semplice \slide ed \`e una parabola. In questo caso, il massimo si
trova all'estremo. Vediamo come opera un algoritmo genetico.

Innanzi tutto occorre trovare un modo efficace di descrivere una
soluzione\slide. A tal fine definisco ci\`o che potrebbe essere l'analogo di un
DNA, che andr\`a a identificare ogni individuo della popolazione di soluzioni.
Dunque ho bisogno di un insieme (chiamato vocabolario) di basi azotate. Nel mio
esempio lo scelgo nel modo pi\`u semplice (e naturale) possibile, ovvero
composto da due elementi: zero e uno. Un DNA consiste in una stringa di cinque
elementi. \`E anche necessaria una codifica dal genotipo al fenotipo, ovvero da
ci\`o che il DNA indica e quale caratteristica effettivamente esprime un certo
individuo. Nella pratica la stringa di zeri e uno verr\`a letta in codice
binario e interpretata come posizione sull'asse delle ascisse.

La popolazione iniziale viene generata in modo totalmente casuale. \slide Ci\`o
significa che ogni possibile DNA ha la stessa probabilit\`a di essere
rappresentato. Si pu\`o pensare in tal senso che ogni gene sia il risultato di
un lancio di una moneta: zero se esce testa e uno se esce croce. Nell'esempio ho
scelto una popolazione molto piccola per poterci lavorare comodamente a mano.
Questi \slide sono i risultati ottenuti.

Giunti a questo punto \`e necessario scegliere gli individui migliori \slide.
Ovvero serve trovare un modo per identificare quali DNA sono i pi\`u adatti a
risolvere il problema. Per questo motivo definisco una funzione, detta di
fitness, che valuta ogni individuo. Nel nostro esempio pi\`u \`e alta l'immagine
di un certo punto, pi\`u il candidato risulta buono, dunque prendere come
funzione di fitness $f(x)$ stessa, \`e una scelta sensata. La tabella di prima
aggiornata risulta dunque essere la seguente\slide.

Adesso si ha tutto l'occorrente per costruire una nuova generazione\slide. Essa
deve avere come proprit\`a innanzi tutto quella di essere mediamente migliore
della precedenti. In secondo luogo deve avere caratteristiche in comune con
essa. La prima fase del passaggio di generazione \`e quello di eliminare gli
individui peggiori. A tal fine definisco la probabilit\`a di sopravvivenza in
questo modo\slide. Si noti che tale probabilit\`a \`e proporzionale al fitness,
e la somma di essa su tutti gli individui \`e pari a uno. Nella pratica
assumeranno i valori che si vedono nella tabella\slide. Risulta immediato
verificare che a moggiore fitness si associa una maggiore probabilit\`a di
sopravvivenza.

A cosa servono? Come passaggio intermedio per arrivare alla generazione
successiva costruisco il cosiddetto \textit{mating pool}\slide, piscina di
accoppiamento. Esso  conterr\`a lo stesso numero di individui della popolazione
originale, e ogni  individuo avr\`a una probabilit\`a pari alla sua
probabilit\`a di sopravvivenza di entrare nel mating pool. In pratica cosa
succede? Per scegliere ogni individuo del mating pool, faccio girare una ruota
simile a quella di una lotteria\slide. Ogni tacca rappresenta  un individuo
della popolazione iniziale. Quindi l'individuo scelto sar\`a esattamente uguale
a quello indicato dalla ruota. Si noti che con questo metodo nel mating pool
potrebbe esserci pi\`u di una copia di uno stesso individuo. Per questo motivo
si rendono necessari i prossimi passaggi di crossing over e di mutazione.\slide

Infatti se non ci fosse n\`e una ricombinazione, n\`e una modificazione dei DNA,
l'algoritmo si limiterebbe a scegliere la migliore soluzione tra quelle
iniziali, che ricordo essere generate casualmente. Serve allora implementare dei
meccanismi che costruiscano soluzioni nuove.

\slide Il primo di questi meccanismi \`e detto, in analogia con ci\`o che avviene
durante la meiosi, crossing over. Nel mating pool, gli individui vengono accoppiati,
e ad ogni coppia succede essenzialmente quello che \`e mostrato in figura: i DNA
si scambiano casualmente alcune sezioni.

\slide Il secondo meccanismo \`e invece detto mutazione. Alcuni








\end{document}

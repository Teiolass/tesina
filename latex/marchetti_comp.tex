\documentclass[a4paper, 11pt, onecolumn]{article}
\usepackage[italian]{babel}
\usepackage[utf8x]{inputenc}
\usepackage{lmodern}
\usepackage{amsmath}
\usepackage[dvipsnames,usenames]{color}

\font\myfont=cmr12 at 25pt
\definecolor{mgray}{gray}{0.3}
\newcommand{\code}{\texttt}
\newcommand*{\MyMarginNoteFormat}{%
    \scriptsize \bfseries \leavevmode \color{mgray}%
}
\newcommand{\margin}[1]{%
    \marginpar
        [\raggedleft  \MyMarginNoteFormat #1]%
        {\raggedright \MyMarginNoteFormat #1}%
}
\newcommand{\slide}{\margin{Slide}}

\title{\myfont Tesina - Algoritmi genetici}
\author{Alessio Marchetti}
\date{}

\begin{document}
\maketitle

\textit{\small \textbf{Nota:} Le note a margine indicano quando cambiare la
slide. Tuttavia alcune slides sono frammentate in pi\`u fasi, e necessitano di
andare avanti anche se non specificato. \vspace{7mm}}

L'argomento della mia tesina \`e uno studio del funzionamento degli algoritmi
genetici. Gli algoritmi genetici sono innanzi tutto algoritmi, ovvero un insieme
di istruzioni utili a risolvere determinate classi di problemi. \slide
L'aggettivo ``genetici'' viene attribuito in quanto i processi che studieremo
attuano meccaniche ispirate ai meccanismi della natura, in particolare quelli
della genetica e della selezione naturale.

La metafora si struttura in questo modo: dato un certo problema, l'obiettivo
dell'algoritmo \`e quello di trovare la soluzione migliore. Allora si \slide
genera in modo casuale una popolazione di candidate soluzioni e di queste si
scelgono le pi\`u adatte a risolvere il problema. Poi  a partire da esse si
costruiscono nuove candidate in modo tale da avere una popolazione mediamente
migliore. Quest'ultimo processo ricalca la riproduzione sessuata, infatti da
coppie di soluzioni, vengono prodotte soluzioni figlie con caratteristiche
comuni ai due genitori.

Per spiegare meglio il funzionamento operativo di un algoritmo genetico, \slide
seguiremo passo a passo un caso specifico. Il problema di cui tratter\`o \`e di
carattere matematico, ovvero la ricerca del massimo e del punto di massimo di
una funzione in un  determinato intervallo chiuso. Quindi, dato un grafico, il
compito \`e quello di cercarne il punto pi\`u alto. La funzione che ho scelto
\`e molto semplice \slide ed \`e una parabola. In questo caso, il massimo si
trova all'estremo. Vediamo come opera un algoritmo genetico.

Innanzi tutto occorre trovare un modo efficace di descrivere una
soluzione\slide. A tal fine definisco ci\`o che potrebbe essere l'analogo di un
DNA, che andr\`a a identificare ogni individuo della popolazione di soluzioni.
Dunque ho bisogno di un insieme (chiamato vocabolario) di basi azotate. Nel mio
esempio lo scelgo nel modo pi\`u semplice (e naturale) possibile, ovvero
composto da due elementi: zero e uno. Un DNA consiste in una stringa di cinque
elementi del vocabolario. \`E anche necessaria una codifica dal genotipo al
fenotipo, ovvero da ci\`o che il DNA indica e quale caratteristica
effettivamente esprime un certo individuo. Nella pratica la stringa di zeri e
uno verr\`a letta in codice binario e interpretata come posizione sull'asse
delle ascisse.

La popolazione iniziale viene generata in modo totalmente casuale. \slide Ci\`o
significa che ogni possibile DNA ha la stessa probabilit\`a di essere
rappresentato. Si pu\`o pensare in tal senso che ogni gene sia il risultato di
un lancio di una moneta: zero se esce testa e uno se esce croce. Nell'esempio ho
scelto una popolazione molto piccola per poterci lavorare comodamente a mano.
Questi \slide sono i risultati ottenuti.

Giunti a questo punto \`e necessario scegliere gli individui migliori \slide.
Ovvero serve trovare un modo per identificare quali DNA sono i pi\`u adatti a
risolvere il problema. Per questo motivo definisco una funzione, detta di
fitness, che valuta ogni individuo. Nel nostro esempio pi\`u \`e alta l'immagine
di un certo punto, pi\`u il candidato risulta buono, dunque prendere come
funzione di fitness $f(x)$ stessa, \`e una scelta sensata. La tabella di prima
aggiornata risulta dunque essere la seguente\slide.

Adesso si ha tutto l'occorrente per costruire una nuova generazione\slide. Essa
deve avere come proprit\`a innanzi tutto quella di essere mediamente migliore
della precedenti. In secondo luogo deve avere caratteristiche in comune con
essa. La prima fase del passaggio di generazione \`e quello di eliminare gli
individui peggiori. A tal fine definisco la probabilit\`a di sopravvivenza in
questo modo\slide. Si noti che tale probabilit\`a \`e proporzionale al fitness,
e la somma di essa su tutti gli individui \`e pari a uno. Nella pratica
assumeranno i valori che si vedono nella tabella\slide. Risulta immediato
verificare che a moggiore fitness si associa una maggiore probabilit\`a di
sopravvivenza.

A cosa servono? Come passaggio intermedio per arrivare alla generazione
successiva costruisco il cosiddetto \textit{mating pool}\slide, piscina di
accoppiamento. Esso  conterr\`a lo stesso numero di individui della popolazione
originale, e ogni  individuo avr\`a una probabilit\`a pari alla sua
probabilit\`a di sopravvivenza di entrare nel mating pool. In pratica cosa
succede? Per scegliere ogni individuo del mating pool, faccio girare una ruota
simile a quella di una lotteria\slide. Ogni tacca rappresenta  un individuo
della popolazione iniziale. Quindi l'individuo scelto sar\`a esattamente uguale
a quello indicato dalla ruota. \slide

Arrivati sin qui \`e lecito chiedersi se effettivamente sia avvenuto un
miglioramento dalla popolazione iniziale. Per dare un analisi quantitativa
occorre definire il fitness medio\slide , che non \`e null'altro che la media
dei fitness di una popolazione. Se ogni individuo ha una probabilit\`a $p$ di
passare nel mating pool, ci si  aspetta che la frazione di individui nel mating
pool uguali ad esso sia proprio $p$. \slide [breve commento alla formula]

Si noti che con questo metodo nel mating pool potrebbe esserci pi\`u di una
copia di uno stesso individuo. Per questo motivo si rendono necessari i prossimi
passaggi di crossing over e di mutazione. Infatti se non ci fosse n\`e una
ricombinazione, n\`e una modificazione dei DNA, l'algoritmo si limiterebbe a
scegliere la migliore soluzione tra quelle iniziali, che ricordo essere generate
casualmente. Serve allora implementare dei meccanismi che costruiscano soluzioni
nuove.

\slide Il primo di questi meccanismi \`e detto, in analogia con ci\`o che
avviene durante la meiosi, crossing over. Nel mating pool, gli individui vengono
accoppiati, e ad ogni coppia succede essenzialmente quello che \`e mostrato in
figura: i DNA si scambiano casualmente alcune sezioni.

\slide Il secondo meccanismo \`e invece detto mutazione. Alcuni geni nel DNA
vengono alterati con una probabilit\`a fissata generalmente bassa, nell'esempio
pari a $0.001$.

\slide Dalla tabella, in cui vengono messi a confronto gli individui delle due
generazioni, risulta effettivamente evidente come si abbia avuto un incremento
generale di fitness. Ci\`o \`e indice del fatto che le soluzioni della
popolazione risolvono meglio il problema posto. Ripetendo pi\`u volte il
percorso seguito di  selezione, crossing over e mutazione, la tendenza sar\`a
quello di una migrazione  e convergenza verso la soluzione migliore. Quante
volte sia necessario ripetere  non \`e stabilito a priori e questo fatto
rappresenta il grosso limite degli algoritmi  genetici. Tuttavia non sempre \`e
necessaria la soluzione migliore in assoluto,  ma solo una buona approssimazione
di essa. Nel video che vi mostrer\`o \margin{Mostrare program max} si riproduce
lo stesso algoritmo seguito ora, solo eseguito dal computer, con pi\`u individui
per generazione e un DNA dalle dimensioni maggiori.

Un altro esempio in cui sono riuscito ad applicare un algoritmo genetico \`e il
problema del commesso viaggiatore. Il problema \`e cos\`i strutturato: dato un
certo insieme di citt\`a (ovvero punti nel piano), trovare il percorso pi\`u
breve  che passi per ciascuno di essi \margin{Mostrare program salesman}. Per
quanto semplice sia la consegna, gli algoritmi  tradizionali sono messi in
difficolt\`a. Sebbene la soluzione possa essere trovata,  richiederebbe un tempo
di elaborazione molto lungo. Un algoritmo genetico magari  non trover\`a la
soluzione migliore, ma almeno una soluzione molto buona, in un tempo minore.

Vediamo ora un confronto pi\`u approfondito tra gli algoritmi genetici e la
selezione naturale vera e propria\slide. La selezione naturale darwiniana si
basa su alcuni semplici presupposti.\slide Il primo \`e quello
dell'ereditariet\`a, ovvero che i genitori trasmettano le proprie
caratteristiche ai propri discendenti. In secondo luogo solo una percentuale
della prole deve riuscire a diventare adulta e procreare a sua volta. La
popolazione inoltre deve presentare un'ampia variet\`a di caratteristiche al suo
interno, e mantenerla con il passare delle  generazioni. Il quarto punto: la
scelta tra quali individui si riprodurranno e  quali no \`e dettata
principalmente dalle interazioni tra l'individuo e l'ambiente. In tal modo le
interazioni favorevoli all'individuo aumenteranno di numero con  il passare del
tempo. Infine i cambiamenti tra gli individui devono essere in grado, dopo un
periodo sufficientemente lungo, di identificare specie differenti.

Per come abbiamo strutturato l'algoritmo, tutti questi punti sono rispettati, se
non l'ultimo. Infatti tra gli individui di una popolazione dell'algoritmo non si
andranno mai a generare le condizioni per la nascita di nuove specie, perch\`e
ogni individuo avr\`a sempre la capacit\`a di riprodursi con ogni altro
individuo, non importa quanto diverso. A un'analisi pi\`u attenta si nota che i
meccanismi naturali per il mantenimento della variabilit\`a genetica  presentano
un'enorme differenza. Infatti la natura ha sviluppato un sistema per  fare in
modo che ogni individuo porti con s\`e molte pi\`u informazioni di quante
effettivamente utilizzi: si tratta della poliploidia. Per esempio l'uomo ha  23
coppie di cromosomi (detti omologhi), di cui per\`o sono utilizzati soltato la
met\`a. Tuttavia nel momento della riproduzione \`e possibile che alla
generazione successiva venga trasferito il cromosoma non utilizzato. I primi
studi su questo genere di fenomeni furono effettuati da Mendel, con i suoi
famosi esperimenti sulle piante di pisello.









\end{document}

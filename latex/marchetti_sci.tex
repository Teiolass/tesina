\documentclass[a4paper, 11pt]{article}
\usepackage[italian]{babel}
\usepackage[utf8x]{inputenc}
\usepackage{lmodern}
\usepackage{amsmath}

\font\myfont=cmr12 at 25pt
\newcommand{\code}{\texttt}

\title{\myfont Tesina - Scienze}
\author{Alessio Marchetti}
\date{}

\begin{document}
\maketitle

Agli inizi del diciannovesimo secolo la teoria pi\`u accreditata per spiegare
l'enorme variet\`a di specie viventi faceva riferimento ad Aristotele e prendeva
il nome di creazionismo. Tra i suoi principali sostenitori si annovera Carl con
Linn\'e (italianizato Carlo Linneo), il quale pensava che gli attuali organismi
fossero stati originati in un momento iniziale, e che da allora le loro
caratteristiche fossero rimaste sostanzialmente inalterate. Tuttavia iniziavano
ad essere presenti alcune teorie che avrebbero poi portato all'attuale teoria
dell'evoluzione. James Hutton formul\`o l'ipotesi dell'attualismo, secondo cui
la terra avrebbe un'et\`a molto grande e che tutti i fenomeni che hanno portato
alla formazione della terra cos\`i come la conosciamo sarebbero tutt'ora in
moto. Jean-Baptiste de Lamark ebbe come tesi che gli animali fossero spinti da
un impulso a salire la scala della complessit\`a, ovvero tendessero a migliorare
le proprie capacit\`a di sopravvivenza. La vera rivoluzione avvenne per\`o con
Charles Darwin, il quale salp\`o a bordo della Beagles verso il Sud America.
Grazie alle osservazioni il naturalista formul\`o una teoria evolutiva. Dal
momento che nella maggior parte degli animali ciascuna coppia di individui \`e
in grado di produrre una prole pi\`u numerosa, il numero di individui in una
popolazione dovrebbe aumentare con andamento esponenziale. Tuttavia ci\`o non
accade, infatti da una coppia  di individui si ha in media una prole di due
indiviui. Il passaggio fondamentale che fece Darwin \`e quello di chiedersi in
quale modo vengono scelti i due che porteranno avanti la specie da una prole
numerosa. La risposta fu che gli individui pi\`u adatti alla sopravvivenza
avranno maggiore probabilit\`a di sopravvivere e riprodursi a loro volta.
Inoltre se si teneva conto del fatto che in generale i figli tendono ad
ereditare caratteristiche dai genitori, allora si hanno conseguenze
interessanti: si prenda ad esempio il famoso esempio delle giraffe.
Probabilmente gli antenati delle attuali giraffe avevano il collo corto e
vivevano in un ambiente con scarsit\`a di cibo, e dunque saper raggiungere le
foglie pi\`u alte degli alberi era importante per la sopravvivenza. Dunque gli
individui con un collo anche di poco superiore alla media erano avvantaggiati
dalla selezione naturale. Si ebbe che da ogni generazione di giraffe si
portassero avanti con maggior frequenza la caratteristica di avere un collo
pi\`u lungo, e, sul lungo termine, iniziarono a comparire gli animali che
conosciamo. 

\end{document}

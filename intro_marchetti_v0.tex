\documentclass[a4paper, 11pt]{article}
\usepackage[italian]{babel}
\usepackage[utf8x]{inputenc}
\usepackage{lmodern}
\usepackage{amsmath}

\font\myfont=cmr12 at 25pt
\newcommand{\code}{\texttt}

\title{\myfont Tesina - Introduzione v.0}
\author{Alessio Marchetti}
\date{}

\begin{document}
\maketitle

Il mio primo programma per computer lo scrissi con l'aiuto di mio padre quando
ancora frequentavo le scuole elementari. Era una banale applicazione in VBA
che determinava se un numero scelto dall'utente fosse pari oppure dispari. Le
istruzioni erano estremamente semplici, ma a partire da quelle poco tempo dopo 
scrissi un altro programma che contava quanti giorni mancassero a Natale. Si
trattava di riuscire a scomporre un problema difficile in piccoli problemini di
cui si conosce la soluzione, e in questo modo il calcolo delle date si riduceva
essenziamlente a tanti problemi di divisibilit\`a.

Con questa tesina mostrer\`o un metodo per risolvere una categoria di problemi
sorprendentemente ampia e variegata. Tale metodo fa utilizzo degli
\textit{algoritmi genetici}. Si tratta di una sorta di adattamento di alcuni
meccanismi del mondo naturale, in particolare quelli della riproduzione e della
selezione naturale, agli ambiti dell'informatica. 

In questo modo problemi estremamente complessi si riducono a trovare gli
individui pi\`u adatti alla sopravvivenza, e alla formazione di nuove
generazioni. Nella pratica il processo fa largo uso di componenti aleatori e
dunque si rende necessaria anche una spiegazione matematica del perch\'e gli
algoritmi genetici funzionino. 

Mi sono approcciato per la prima volta agli algoritmi genetici un paio di anni
fa, grazie al libro ``\textit{The nature of the code}'' di Daniel Shiffman. In
tale testo ho potuto trovare numerosi esempi scritti in modo molto semplice che
mi hanno introdotto all'argomento e allo stesso tempo incuriosito. 


\end{document}

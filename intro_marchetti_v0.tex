\documentclass[a4paper, 11pt]{article}
\usepackage[italian]{babel}
\usepackage[utf8x]{inputenc}
\usepackage{lmodern}
\usepackage{amsmath}

\font\myfont=cmr12 at 25pt
\newcommand{\code}{\texttt}

\title{\myfont Tesina - Introduzione v.0}
\author{Alessio Marchetti}
\date{}

\begin{document}
\maketitle

Risulta superfluo oltre che banale affermare che lo sviluppo tecnologico abbia
fatto passi da gigante negli ultimi decenni. E se da una parte il miglioramento
dell'hardware \`e evidente, con dispositivi sempre pi\`u piccoli, sempre pi\`u
potenti, da un'altra vi \`e un miglioramento del software, tratta solamente
della capacit\`a di estendere vecchi algoritmi su nuovi processori. Il risultato
forse che ha influito maggiormente sulle nostre vite in tempi recenti (e
che probabilmente continuer\`a ad espandere il suo raggio d'azione) \`e quello
dello sviluppo di intelligenze artificiali. Dai riconoscimenti facciali, alle
auto che si guidano da sole, dagli \textit{assistant} alle AI dei videogames, 


\end{document}
